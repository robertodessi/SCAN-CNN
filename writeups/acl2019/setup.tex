\section{SCAN dataset and models configuration}
\label{sec:setup}

The SCAN dataset which was introduced in \cite{Lake:Baroni:2017} tests generalization capabilities of a learning model.
It is designed as a translation task where a sequence of spatial navigation commands have to be translated into corresponding 
spatial navigation actions. [\dots] 

Convolutional models have proved to be efficient and powerful solutions [\dots] \cite{kalchbrenner:etal:2016, gehring:etal:2017}.
Self-attention networks have also gained attention recently as they were able to outperform convolutional models on different
machine translation benchmarks \cite{chen:etal:2018,vaswani:etal:2017}.
Our work is focused on assessing whether a powerful and efficient neural network based on convolutional connections
is capable of learning to compose and generalize to unseen meanings when carefully trained on smaller semantic blocks.
We decided to make the focus of our work the qualitative and quantitative study of a convolutional architecture, as
it is be easier to inspect its inner components compared to self attention based networks.
We leave for future work similar comparative study that employs self-attentive models.
Moreover, we consider the use of convolutional filters as an interesting dimension to inspect, especially when considering 
its ability to possibly assess the small and self contained semantics of words within the larger context of the sentence [\dots make a proper example with SCAN].
The desidered systematic compositionality that we seek to find in modern neural networks could indeed be described as the
human ability to extract, process and combine small unit that are semantically meaningful, in order to create a larger and again meaningful semantic block.
The hypothesis of the use of convolutions as an important parameter for the ability of the network
to correctly process semantic compositionality is investigated in our second experiment presented in section \ref{subsec:exp2}.

The model used in our experiments is the convolutional sequence to sequence neural network introduced in \cite{gehring:etal:2017}.
This model uses convolutional filters and Gated Linear Units \cite{dauphin:etal:2016} along with a Multi-step attention mechanism.
The attention module computes different steps (or computations) before calculating its final attention scores at any given timestep, similarily 
to the multi-hop mechanism proposed in \cite{sukhbaatar:etal:2105}.
[\dots]
