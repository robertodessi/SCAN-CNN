\section{Introduction}
\label{sec:intro}

The advent of powerful deep neural network has rekindled classic
debates on the natural language processing abilities of such models
\cite[e.g.,][]{Kirov:Cotterell:2018,Linzen:etal:2018,McCoy:etal:2018,Pater:2018}. \citet{Lake:Baroni:2017}
and \citet{Loula:etal:2018} have proposed the SCAN challenge to
directly assess the ability of sequence-to-sequence networks to
perform systematic, compositional generalization of linguistic
rules. Their results, and those of \citet{Bastings:etal:2018}, have
shown that modern recurrent networks (gated RNNs, such as LSTMs and GRUs) generalize well
to new sequences that resemble those encountered in training,
but achieve very low performance when generalization must be
supported by a systematic compositional rule, such as ``to X twice
means to X and then to X again''.\footnote{\citet{Bastings:etal:2018}
  report higher performance for accurately tuned recurrent models on
  some of the less challenging SCAN tasks, but still very low accuracy on the \emph{jump} split we discuss below.}

Non-recurrent models, such as convolutional neural networks
\cite[CNNs,][]{kalchbrenner:etal:2016, gehring:etal:2016,
  gehring:etal:2017} and self-attentive models
\cite{vaswani:etal:2017, chen:etal:2018} have recently reached
comparable or better performance than recurrent networks on machine
translation and other sequence-to-sequence benchmarks. Their
linguistic properties are however  still generally poorly
understood. \newcite{Tang:etal:2018} have shown that RNNs and
self-attentive models are better tan CNNs at capturing long-distance
agreement, while self-attentive networks in particular excel at
word sense disambiguation. In an extensive and systematic comparison,
\newcite{Bai:etal:2018} showed CNNs to be better across the board
than RNNs, although in general the differences are not huge.

In this paper, we evaluate an out-of-the-box CNN model on the most
challenging SCAN tasks, and we uncover the surprising fact that
\emph{CNNs are dramatically better than RNNs at compositional
  generalization}. As they are more cumbersome to
train and adapt, we leave testing of self-attentive networks to future
work.

% Convolutional models have recently proved to be efficient and powerful solutions in machine translation 
% \cite{kalchbrenner:etal:2016, gehring:etal:2016, gehring:etal:2017}.
% The lack of recurrent connections makes their training faster and more parallelizable 
% Self-attention networks have also gained attention recently as they were able to outperform convolutional models on different
% machine translation benchmarks \cite{vaswani:etal:2017, chen:etal:2018} while showing a comparable training cost.
% Our work is focused on assessing whether a powerful and efficient neural network based on convolutional connections
% is capable of learning to compose and generalize to unseen meanings when carefully trained on smaller semantic units.
% We decided to make the focus of our work the qualitative and quantitative study of a convolutional architecture, as
% it is be easier to inspect its inner components compared to self attention based networks.
% We leave for future work similar comparative study that employs self-attentive models.



