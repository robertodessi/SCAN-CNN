\section{Introduction}
\label{sec:intro}

Recent deep neural network successes rekindled classic
debates on their natural language processing abilities
\cite[e.g.,][]{Kirov:Cotterell:2018,McCoy:etal:2018,Pater:2018}. \citet{Lake:Baroni:2017}
and \citet{Loula:etal:2018} proposed the SCAN challenge to
directly assess the ability of sequence-to-sequence networks to
perform systematic, compositional generalization of linguistic
rules. Their results, and those of \citet{Bastings:etal:2018}, have
shown that modern recurrent networks (gated RNNs, such as LSTMs and GRUs) generalize well
to new sequences that resemble those encountered in training,
but achieve very low performance when generalization must be
supported by a systematic compositional rule, such as ``to X twice 
you X and X''(e.g., to jump twice, you jump and jump again).

% \footnote{\citet{Bastings:etal:2018}
  % report higher performance for accurately tuned RNNs on
  % some of the less challenging SCAN tasks.}

Non-recurrent models, such as convolutional neural networks
\cite[CNNs,][]{kalchbrenner:etal:2016, gehring:etal:2016,
  gehring:etal:2017} and self-attentive models
\cite{vaswani:etal:2017, chen:etal:2018} have recently reached
comparable or better performance than RNNs on machine translation and
other benchmarks. Their linguistic properties are however still
generally poorly understood. \newcite{Tang:etal:2018} have shown that
RNNs and self-attentive models are better than CNNs at capturing
long-distance agreement, while self-attentive networks excel at word
sense disambiguation. In an extensive comparison,
\newcite{Bai:etal:2018} showed that CNNs generally outperform RNNs,
although the differences were typically not huge. We evaluate here an out-of-the-box CNN on the most
challenging SCAN tasks, and we uncover the surprising fact that
\emph{CNNs are dramatically better than RNNs at compositional
  generalization}. As they are more cumbersome to train, we leave
testing of self-attentive networks to future work.

% Convolutional models have recently proved to be efficient and powerful solutions in machine translation 
% \cite{kalchbrenner:etal:2016, gehring:etal:2016, gehring:etal:2017}.
% The lack of recurrent connections makes their training faster and more parallelizable 
% Self-attention networks have also gained attention recently as they were able to outperform convolutional models on different
% machine translation benchmarks \cite{vaswani:etal:2017, chen:etal:2018} while showing a comparable training cost.
% Our work is focused on assessing whether a powerful and efficient neural network based on convolutional connections
% is capable of learning to compose and generalize to unseen meanings when carefully trained on smaller semantic units.
% We decided to make the focus of our work the qualitative and quantitative study of a convolutional architecture, as
% it is be easier to inspect its inner components compared to self attention based networks.
% We leave for future work similar comparative study that employs self-attentive models.



