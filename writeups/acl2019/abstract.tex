\begin{abstract}
  \citet{Lake:Baroni:2017} introduced the SCAN dataset to probe the
  ability of seq2seq models to capture compositional generalizations,
  such as those that allow humans to infer the meaning of \emph{``jump
    around''} from the component words, even if they have never heard
  this exact phrase. Lake and Baroni and following researchers focused
  their evaluation on recurrent networks (RNNs), finding that they
  completely fail the most systematic generalization cases. We tested
  an out-of-the-box convolutional network (CNN) on such tasks, and we
  report hugely improved performance with respect to RNNs. Error analysis shows that, despite this big improvement, the CNN
  does not seem to follow systematic rules, suggesting that the
  difference between compositional and non-compositional behaviour is
  less clear-cut than typically assumed.
  % The ability to systematically compose new utterances from smaller and known units is a
  % key abilty for humans, who are able to master such property withouth being instructed to do so.
  % %Conversely, modern neural networks artchitectures are unable to perform even basic
  % %compositional operation on unseen words.
  % Recurrent neural network models are good at generalizing when they are trained on examples 
  % that are similar in structure and length to the ones seen at test time, but their performance
  % decrease if length sequence is different between train and test data, finally showing poor performance
  % when they have to generalize to an entirely new composed sequence.
  % In this paper we report a set of experimental results of several 
  % configurations of convolutional neural networks (CNNs)
  % on the SCAN dataset \cite{Lake:Baroni:2017}. We find that CNNs largely outperform recurrent neural networks
  % and are able to achieve successful performance even on the hardest split of SCAN.
  % %Systematic compositionality as defined by Lake and Baroni, 2017,
  %is the "the algebraic capacity to understand and produce a
  %potentially infinite number of novel combinations from known components" 
\end{abstract}
